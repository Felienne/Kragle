\documentclass{sig-alternate}

\usepackage{url}
\usepackage{graphicx}

\newcommand{\nPrograms}{250.166}
\newcommand{\nemptyPrograms}{14.285}
\newcommand{\nScriptPrograms}{233.514}
\newcommand{\nProgramsWithClones}{67.177}
\newcommand{\dup}{\emph{Duplication}~}

\newcommand{\todo}[1]{\textbf{#1}}

\begin{document}
%
% --- Author Metadata here ---
%\conferenceinfo{WOODSTOCK}{'97 El Paso, Texas USA}
% --- End of Author Metadata ---

\title{How kids code and how we know it:\\An exploratory study in the Scratch project repository }

\numberofauthors{1}
\author{
% 1st. author
\alignauthor
(anonymized submission)
%Ben Trovato\titlenote{Dr.~Trovato insisted his name be first.}\\
%       \affaddr{Institute for Clarity in Documentation}\\
%       \affaddr{1932 Wallamaloo Lane}\\
%       \affaddr{Wallamaloo, New Zealand}\\
%       \email{trovato@corporation.com}
% 2nd. author
%\alignauthor
%G.K.M. Tobin\titlenote{The secretary disavows
%any knowledge of this author's actions.}\\
%       \affaddr{Institute for Clarity in Documentation}\\
%       \affaddr{P.O. Box 1212}\\
%       \affaddr{Dublin, Ohio 43017-6221}\\
%       \email{webmaster@marysville-ohio.com}
}

%\date{30 July 1999}

\maketitle
\begin{abstract}
Block-based programming languages like Scratch, Alice and Blockly are becoming increasingly common as introductory languages in programming education. There is substantial research showing that those visual programming environments are suitable for teaching programming concepts. In this paper we explore the characteristics of Scratch programs. To that end we have scraped the Scratch public repository and retrieved 250.000 projects. We present an analysis of those projects in terms of complexity, used abstractions and programming concepts, and coding style. We found that programming abstraction concepts like functions are not commonly used. We further investigate the presence of code smells and bad programming practices, including code duplication, dead code, long method and large class smells. Our findings indicate that Scratch programs suffer from code smells and especially code duplication.
\end{abstract}

% A category with the (minimum) three required fields
%\category{H.4}{Information Systems Applications}{Miscellaneous}
%A category including the fourth, optional field follows...
%\category{D.2.8}{Software Engineering}{Metrics}[complexity measures, performance measures]

\terms{terms}

\keywords{Scratch, programming practices, code smells, static analysis}

\section{Introduction}

Scratch is a programming language developed to teach children programming by enabling them to create games and interactive animations. The public repository of Scratch programs contains over 12 million projects. Scratch is a \emph{block-based} language: users manipulate blocks to program. Block-based languages are visual languages, but also use some successful aspects of text-based languages such as limited text-entry and indentation, and as such are closer to `real', textual programming than other forms of visual programming, like dataflow languages are.

Block-based languages have existed since the eighties, but have recently found adoption as tools for programming education. In addition to Scratch, also Alice~\cite{conway_alice:_1994}, Blockly\footnote{\url{https://developers.google.com/blockly/}} and App Inventor~\cite{wolber_app_2011} are block-languages aimed at novice programmers.

In this paper we explore code clones smells in the context of block-based languages. \todo{hier nog wat background over clones, zie ICSE `13 uiteraard}

Clones so far have mostly been studied in the contact of object-oriented, textual source code, but some other directions have been taken, including spreadsheets and SimuLink programs. 

% Recently, it has been demonstrated that code smells occur in various alternative programming environments including spreadsheets~\cite{hermans_detecting_2012, hermans_detecting_2014}, Yahoo! Pipes~\cite{stolee_refactoring_2011} and LabView~\cite{chambers_smell-driven_2013}.

We ourselves have recently performed an experiment where we compared the performance of novice Scratch programmers between three 

Knowing that code clones can be harmful, we now turn to the question of whether they are common. As such, the goal of this paper is to \textbf{investigate whether code clones are common in public Scratch programs}.  To address this goal we have obtained \nPrograms~public Scratch programs by scraping the list of recent programs \todo{link}. We investigate three types of clones: clones of entire scripts within sprites, clones of entire between sprites, and cloned conditions.

The results show that \todo{todo}

The contributions of this paper are as follows:

\begin{itemize}
	\item{A public data set of \nScriptPrograms~Scratch programs (Section \ref{dataset})}
	\item{A clone Scratch detection algorithm with an open source implementation (Section \ref{algo})}
	\item{An empirical evaluation of the occurrence of three types of clones in Scratch programs (Section \ref{results})}
\end{itemize}

Research questions?

\section{Background and Motivation}
\label{sec:background}
Block-based languages go back to 1986, when Glinert introduced the BLOX language~\cite{e._glinert_towards_1986}. BLOX consists of puzzle-like programming statements that can be combined into programs by combining them both vertically and horizontally. After a decade of little activity into block-based languages, they became a research topic again, starting with Alice~\cite{conway_alice:_1994}. More recently, new block-based languages have gained widespread popularity, especially powered by Scratch~\cite{resnick_scratch:_2009} and Blockly\footnote{\url{https://developers.google.com/blockly/}}. Over 100 million students have tried Blockly via Code.org, and the Scratch repository currently hosts over 12 million projects. Unlike in BLOX, in these new block-based languages the programming blocks can only be combined vertically, resembling textual code more.

Since their introduction, studies have demonstrated the applicability of block-based languages as a tool for education. Scratch, for example, was evaluated with a two-hour introductory programming curriculum for 46 subjects aged 14~\cite{meerbaum-salant_learning_2010}. This study indicated that Scratch could be used to teach computer science concepts: analysis of the pre- and post-tests showed a significant improvement after the Scratch course, although some concepts like variables and concurrency remained hard for students.

Moskal \emph{et al.}~\cite{b._moskal_evaluating_2005} compared computer science students who studied Alice before or during their first programming course to students that only took the introductory computer science course. Their results show that exposure to Alice significantly improved students' grades in the course, and their retention in computer science in general over a two year period. A follow-up study by Cooper \emph{et al.}~\cite{cooper_teaching_2003} obtained similar results, showing that a curriculum in Alice resulted in improved grades and higher retention in computer science.

Most convincingly, Price and Barnes performed a controlled experiment in which students were randomly assigned to either a text-based or a block-based interface in  which they had to perform small programming tasks~\cite{price_comparing_2015}. Their experiment showed that students in the block-based interface were more focused and completed more of the activity's goals in less time.

Summarizing the above, we conclude that block-based languages have a clear potential to be a great tool for introductory programming education, in some cases even outperforming text-based languages.

\section{Relevant Scratch Concepts}
\label{sec:scratch}
This paper is by no means an introduction into Scratch programming, we refer the reader to \cite{brennan_creative_2014} for an extensive overview. To make this paper self-contained, however, we explain a number of relevant concepts in this section. 

Scratch is a block-based programming language aimed at children, developed by MIT. Scratch can be used to create games and interactive animations, and is available both as a stand-alone application and as a web application. Figure \ref{fig:ui} shows the Scratch user interface in the Chrome browser.

\subsection{Sprites}
Scratch code is organized by `sprites': two-dimensional pictures each having their own associated code. Scratch allows users to bring their sprites to life in various ways, for example by moving them in the plane, having them say or think words or sentences via text balloons, but also by having them make sounds, grow, shrink and switch costumes. The Scratch program in Figure \ref{fig:ui}\footnote{\url{https://scratch.mit.edu/projects/97086781/}} consists of one sprite, the cat, which is Scratch's default sprite and logo. The code in the sprite will cause the cat to jump up, say ``hello'', and come back down, when the green flag is clicked, and to make the `meow' sound when the space bar is pressed.

\begin{figure}
	\begin{center}
		\includegraphics[width=\columnwidth]{fig/ui.png}
		\caption{The Scratch user interface consisting of the `cat' sprite on the left, the toolbox with available blocks in the category `motion' in the middle and the code associated with the sprite on the right. The upper right corner shows the actual location of the sprite.}
		\label{fig:ui}
	\end{center}
\end{figure} 

\subsection{Signals}
\todo{uitleggen wat een signal is en evt ook al hoe het een kloon vervangt?}

\subsection{Events}
Scratch is \emph{event-driven}: all motions, sounds and changes in the looks of sprites are initiated by events. The canonical event is the `when Green Flag clicked', activated by clicking the green flag at the top of the user interface. In addition to the green flag, there are a number of other events possible, including key presses, mouse clicks and input from a computer's microphone or webcam. In the Scratch code in Figure \ref{fig:ui} there are two events: `when Green Flag clicked' and `when space key pressed'.

\subsection{Scripts}
Source code within sprites is organized in scripts: a script always starts with an event, followed by a number of blocks. The Scratch code in Figure \ref{fig:ui} has two distinct scripts, one started by clicking on the green flag and one by pressing the space bar. It is possible for a single sprite to have multiple scripts initiated by the same event. In that case, all scripts will be executed simultaneously. For example, the code on the left of Figure \ref{fig:comparison} has five scripts associated with the `when Green Flag clicked' event.

\subsection{Remixing}
Scratch programs can be shared by their creators in the global Scratch repository\footnote{\url{https://scratch.mit.edu/explore/projects/all/}}. Shared Scratch programs can be `remixed' by other Scratch users, which means that a copy of this program is placed in the user's own project collection, and can be then further changed. The `remix tree' of projects is public, so users can track which users remix their programs, a bit similar forking in GitHub. Contrary to forking though, changes upstream cannot be integrated back into the original project.


\begin{figure}
	\begin{center}
		\includegraphics[width=6cm]{fig/makeBlockSmall.png}
		\caption{}
		\label{fig:makeBlock}
	\end{center}
\end{figure} 

\begin{figure}
	\begin{center}
		\includegraphics[width=6cm]{fig/useBlockSmall.png}
		\caption{}
		\label{fig:useBlock}
	\end{center}
\end{figure} 




\section{Dataset}
\label{dataset}
We scraped all Scratch programs uploaded \todo{add date range?} from the Scratch website with a scraping program. This resulted in he JSON code for \nPrograms~Scratch programs. In addition to the programs themselves, we also gathered metadata including the numbers of views, loves, favorites and remixes. Out of the \nPrograms~, we failed to analyze 2.367 programs due to various technical difficulties \todo{add details?} 

\subsection{Empty programs}

Also, interestingly enough, \nemptyPrograms~were empty (5.7\%), at least in terms of scripts. So Scratch users were sharing program already that did not contain code yet. In some cases \todo{analyze too?} they contained scripts and costumes but no code and in others they were entirety empty apart from the Scratch cat added by default.

\subsection{Block Usage}

\subsection{Block Size}

\subsection{Dead code}

\subsection{User-defined Blocks}

\subsection{Remixing}

\section{Clones Detection for Scratch programs}
\label{algo}
\label{clone_Detection}
\todo{How it works, Fenia, I can write this if you want}


\section{Results}
\label{results}
The overarching research question of this work is how common Scratch users clone in their programs. As explained in Section \ref{clone_Detection} we distinguish three different forms of clones: Exact clones within one sprite, exact clones between sprites of one program, clones of `wait' blocks. This section presents the results of our clone detection algorithm on the \nScriptPrograms~ with scripts.

In total \nProgramsWithClones~(26.9\% of programs with scripts) exhibit one of the three types of cloning.The most common type of cloning is copying entire scripts between sprites. This occurs in 62.939 programs (25.2\%). In 8.220 programs (3.3\%) wait conditions are cloned and in 2.463 files there is a script cloned within a sprite.

\subsection{Exact clones within one sprite}
In total, we found \todo{x} cases of exact script clones within one sprite. Upon further inspection, we found that some of them consisted of unconnected blocks, as depicted in Figure \ref{fig:Unconnected_clones}. While this is technically a clone, as this is \emph{dead code} and will never be executed, this could be considered unharmful sketching. Looking only at clones starting with an event block, \todo{...} clones remain over \todo{?} files.

\begin{figure}
	\begin{center}
		\includegraphics[width=6cm]{fig/Unconnected_clones.png}
		\caption{Cloned blocks that are not connected to an event. Program id: 12237615}
		\label{fig:Unconnected_clones}
	\end{center}
\end{figure} 


\subsection{Exact clones between sprites of one program}

\subsection{Clones of `wait' blocks}
Previous work has established that the cloned `wait' blocks can be harmful when modifying Scratch code, which is the reason why we investigated this particular type of cloning. As outlined in section \todo{?} signals can replaced repeated wait blocks, by catching the condition in one place and broadcasting a signal to all other sprites. This is why we found it interesting to see in how many of the cases where a wait block was cloned, the creator of the program also used signals in their program. This indicates that the concept of signals is known, and hence the duplicate condition is not used because the user is not aware of the alternative. In \todo{...} of the cases the program with the duplicated wait block also contained one or more signals. 

\subsection{Conclusion}

\subsection{Implications}
Our results have some interesting implications for designers of educational programming languages. 

\subsubsection{Unconnected Blocks}
In the whole dataset, not just in the programs with clones, we observe unconnected scripts, scripts without an event block and events without associated code. This is a form of \emph{dead code}, code that will never be executed. Because this dead code caused visual clutter, it would be better to have a separate workspace (much like the `backpack in Scratch' where users can store blocks temporarily and use them later. A programming interface could ask actively encourage users to move unconnected block when they exit the environment to encourage a `clean' working space. 

\subsubsection{Code Clones Between Sprites}
With occurrences in one quarter of the 250 thousand programs, the use of exactly identical clones between sprites is extremely common. In a sense, the Scratch users are not to blame here, Scratch does not support the sharing of \todo{right word? procedures} between scripts, only within them. So in many cases there is no other way to share the functionality than making a copy. We are not aware of the underlying rationale of the Scratch team that lead to this decision, however it seems that a large part of the Scratch users needs this functionality. 

\subsubsection{Refactoring support for duplicate wait blocks}
In todo{...} of the cases, the programs with duplicate conditions, also signals were present. This means that it is feas

\section{Discussion}




\section{Related Work}
\label{sec:related}

\subsection{Code Smells}
Efforts related to our research include works on code smells, initiated by the work by Fowler~\cite{fowler_refactoring:_1999}. His book gives an overview of code smells and corresponding refactorings. Fowler's work was followed by efforts focused on the automatic identification of code smells by means of metrics. Marinescu~\cite{marinescu_detecting_2001} for instance, uses metrics to identify \emph{suspect} classes: classes which could have design flaws. Lanza and Marinescu~\cite{lanza_object-oriented_2006} explain this methodology in more detail. Alves \emph{et al.}~\cite{alves_deriving_2010} focus on a strategy to obtain thresholds for metrics from a benchmark. Olbrich \emph{et al.} furthermore investigates the changes in smells over time, and discusses their impact~\cite{olbrich_evolution_2009}. Moha \emph{et al.}~\cite{moha_decor:_2010} designed the `DECOR' method which automatically generates a smell detection algorithms from specifications. The CCFinder tool~\cite{kamiya_ccfinder:_2002} finally, aims at detecting clones in source code, which are similar to our \dup smell.

\subsection{Smells beyond the OO paradigm}
In recent times, code smells have been applied to programs outside of the regular programming domain. In our past work, we have, for example, studied code smells within spreadsheets, both at the formula level~\cite{hermans_detecting_2014} and between worksheets~\cite{hermans_detecting_2012}.

More recently, we compared two datasets: one containing spreadsheets which users found unmaintainable, and a version of the same spreadsheets rebuilt by professional spreadsheet developers. The results show that the improved versions suffered from smells to a lesser degree, increasing our confidence that presence of smells indeed coincides with users finding spreadsheets hard to maintain~\cite{jansen_code_2015}.

In addition to spreadsheets, code smells have also been studied in the context of Yahoo! Pipes a web mashup tool. An experiment demonstrated that users preferred the non-smelly versions of Yahoo Pipes programs~\cite{stolee_refactoring_2011}. 

%\subsection{Agile practices in education}
%A third related line of work are research directions where other methods from agile development, like testing and refactoring, have been applied in education. For example, the use of test-driven development has been studied within the context of programming education at university level, using text-based languages in several studies. Most controlled experiments resulted in increased code quality for the group of students using TDD, for example [13] reported on a 35\% reduction in defects, while [14] found 45\%. For an extensive overview of TDD approaches in university programming courses, see [15]. \todo{referenties fixen}

\subsection{Quality of Scratch programs}
Finally, there are other works on the quality of Scratch programs. There is for example the Hairball Scratch extension~\cite{boe_hairball:_2013}, which is a lint-like static analysis tool for Scratch that can detect, for example, unmatched broadcast and receive blocks, infinite loops and duplication. An evaluation of 100 Scratch programs showed that Scratch programs indeed suffer from duplication and bad naming~\cite{moreno_automatic_2014}.

Most related to our study is the work by Moreno \emph{et al.}~\cite{moreno-leon_dr._2015} who gave automated feedback on Scratch programs to 100 children aged 10 to 14. Their results demonstrated that feedback on code quality helped improve students' programming skills.


\bibliographystyle{abbrv}
\bibliography{zotero}
\end{document}
